% !TeX root = RJwrapper.tex
\title{nlsCompare: An R package for comparing nonlinear least squares
packages in R}
\author{by Arkajyoti Bhattacharjee\(^1\), John C. Nash\(^2\), and Heather
Turner\(^3\)}

\maketitle

\abstract{%
Nonlinear least squares (nls) is extensively used in nonlinear modeling.
This paper introduces the package \textbf{nlsCompare} that attempts to
compare the best values of parameters and sum-of-squares in nls problems
with existing tools for nls like \texttt{nls}(\textbf{base}),
\texttt{nlxb} (\textbf{nlsr}), \texttt{nlsLM}(\textbf{minpack.lm}) and a
new experimental package \textbf{nlsj} developed by the authors as a
product of Google Summer of Code (GSoC), 2021.
}

\hypertarget{introduction}{%
\subsection{Introduction}\label{introduction}}

While working on the GSoC project \textbf{Improvements to nls()} under
The R Project for Statistical Computing, the authors realized that to be
able to observe improvements made via experimental packages such as
\texttt{nlsj}, developers needed to quickly compare results with the
best known reference values for each of a battery of problems.

\textbf{nlsCompare}:

\begin{itemize}
\tightlist
\item
  has a spreadsheet with pointers to test problem files along with
  reference solutions. These problem/cases are based on packages like
  \textbf{NRAIA} with data, models, constraints and starting values.
\item
  has a spreadsheet containing the methods (solvers, algorithms, control
  settings) to be tested
\item
  generates easy-to-read spreadsheets that compare results (sum of
  squared residuals and model parameters with the best reference values
  yet observed. Since there can be multiple methods, these spreadsheets
  allow for fairly rapid comparisons. Moreover, the columns can be
  sorted to allow specific features to be compared more easily.
\item
  gives appropriate messages to assist the understanding of errors or
  where to do improvements ?? does it do this.
\end{itemize}

\textbf{nlscompare} draws inspiration from R's testing library
\textbf{testthat} and uses ideas from packages like \textbf{benchmarkme}
to identify the computing environment. Currently, we have restricted our
attention to the R computing environment.

\hypertarget{need-for-nlscompare}{%
\subsection{Need for nlsCompare}\label{need-for-nlscompare}}

Regression testing is useful not only to validate correctness of a
program but often also for tracking the quality of its output. While
working on improving \texttt{nls}, author\(^2\) noted that to compare
the improvements made over \texttt{nls} via \textbf{nlsj}, unit testing
via \textbf{testthat} is not enough, largely because some desirable
features such as bounds constraints within the default Gauss-Newton
method is not provided. That is, we are trying to provide for a result
currently unavailable in \texttt{nls()}. While we did for some time
pursue the use of other nls functions, we eventually moved to using
reference values from our problems spreadsheet. ?? needed?? To be able
to add more improvements to \textbf{nlsj}, it is required to compare it
with other existing packages' nls functions. ?? but keep this?? This
would allow us to note which methods are supported by a particular
function that are not in another and to check the consistency of
supposedly equivalent results from different functions.

\textbf{nlscompare} attempts to achieve the above, outputting a simple
informative spreadsheet that indicates whether an existing or new
package's function provides consistent results with the current best
obseved values. Further, it outputs an error log that indicates whether
a solver (viz.~\texttt{nls} or \texttt{nlsLM}, etc.) fail in a
particular problem with particular methods (algorithms, starting points,
etc.)

\hypertarget{workflow}{%
\subsection{Workflow}\label{workflow}}

The \textbf{nlscompare} needs to be run in the following hierarchical
steps -

\begin{enumerate}
\def\labelenumi{\arabic{enumi}.}
\tightlist
\item
  \texttt{setup\_dir()}: choose and set the working directory save final
  csv outputs.
\item
  \texttt{machineId()}: create the machine ID to be used in the output
  csv files.
\item
  \texttt{create\_db()}: create dataframe to store database of nls
  problems
\item
  \texttt{create\_elog()}: create dataframe to store error-log in
  solving the nls problems
\item
  \texttt{csv\_exists}: check if the csv files \textbf{nlsDatabase.csv}
  and \textbf{nlsErrorLog.csv} exist already in the directory chosen in
  1. Create them if not present
\item
  \texttt{run()}: this uses the problem names - \textbf{Name} , the
  reference values - \textbf{nPars}, \textbf{Parameters} and
  \textbf{ssquares} from \textbf{problems.csv} , the methods from
  \textbf{methods.csv}, does the comparisons and error logging and edits
  the dataframes accordingly.
\item
  \texttt{write\_csvs}: write the dataframes returned by \texttt{run}
  into \textbf{nlsDatabase.csv} and \textbf{nlsErrorLog.csv}
\item
  \texttt{rm\_nls} : delete the global variables created due to running
  of the above functions
\end{enumerate}

A detailed explanation on how to install and use the package is
available in the package vignette -
\href{https://github.com/ArkaB-DS/nlsCompare/blob/master/inst/docs/nlsCompare-Usage.pdf}{nlsCompare:
How to Use it?}

\hypertarget{package-content-details}{%
\subsection{Package content details}\label{package-content-details}}

\hypertarget{methods.csv}{%
\subsubsection{methods.csv}\label{methods.csv}}

This csv file contains three columns:

\begin{enumerate}
\def\labelenumi{\arabic{enumi}.}
\tightlist
\item
  \textbf{solver:} This is the name of the function to be used to solve
  the nls problem. Currently, it supports \texttt{nlsj::nlsj},
  \texttt{nls}, \texttt{nlsr::nlxb} and \texttt{minpack.lm::nlsLM}.
\item
  \textbf{algorithm:} This is the name of the algorithm used. Currently,
  supports \texttt{default}, \texttt{plinear}, \texttt{port},
  \texttt{LM} and \texttt{marquardt}, but only when these are
  appropriate to a given solver.
\item
  \textbf{control:} This contains the package function and its
  associated control settings to control the iterations or other
  features of the solver and/or problem.
\end{enumerate}

The \textbf{methods.csv} file is essentially a list of detailed methods
for use inside the \texttt{run} function.

Users can add new solvers, algorithms and controls as in the existing
csv file.

\hypertarget{problems.csv}{%
\subsubsection{problems.csv}\label{problems.csv}}

This csv file contains four sets of columns:

\begin{itemize}
\tightlist
\item
  \textbf{Name:} This contains the names of files present in the
  \textbf{./nlsCompare/inst/scripts} of the package and is used by the
  \texttt{run()} function to source the \textbf{Name}-ed files inside
  the function.
\item
  \textbf{ssquares:} This contains the best (least) possible sum of
  squares achievable in the nls problem as present in the
  \textbf{Name}-ed file.
\item
  \textbf{nPars:} The number of parameters in the nls problem present in
  a \textbf{Name}-ed file.
\item
  \textbf{Pars:} This and the adjacent (\textbf{nPars} - 1) columns
  contain the best parameter values of the nls problem present in the
  \textbf{Name}-ed file.
\end{itemize}

The ``best'' values are based on the observations made by the authors on
using different nls packages. These ``best'' values serve as a reference
that is used inside the \texttt{run} function to compare with a
\textbf{solver}'s output using the corresponding \textbf{algorithm} and
\textbf{control}. In case a user finds a ``better'' (see below) value,
the values inside this csv should be changed manually by the user.

\hypertarget{nlscompareinstscripts}{%
\subsubsection{./nlsCompare/inst/scripts}\label{nlscompareinstscripts}}

This directory contains the R scripts for numerous problems in nls that
are sourced inside the \texttt{run} function. Each file has a naming
structure: ``problemName\_x'' where x is a integer number. ``1'' means
that it is a parent test file of the ``problemName'' family of test
files. The parent file mainly contains -

\begin{itemize}
\tightlist
\item
  \texttt{NLSdata}: A dataframe that contains the data
\item
  \texttt{NLSformula}: nonlinear model formula
\item
  \texttt{NLSstart}: Starting values for the nls problem
\item
  \texttt{NLSlower}: An lower bound for the nls problem
\item
  \texttt{NLSupper}: An upper bound for the nls problem
\item
  \texttt{NLSweights}: weights for the nls problem
\item
  \texttt{NLSsubset}: for subsetting the dataframe
\item
  \texttt{NLSmethods}: a dataframe that stores the contents of
  \textbf{methods.csv}
\item
  \texttt{NLSproblems}: a dataframe that stores the contents of
  \textbf{problems.csv}
\item
  \texttt{refsol}: the package name that gives the best results to an
  nls problem listed in \textbf{problems.csv}
\item
  \texttt{NLSssquares}: the sum of squares produced by \textbf{refsol}.
  This is derived from \textbf{problems.csv}'s \textbf{ssquares} column
\item
  \texttt{NLSpars}: the parameters produced by \textbf{refsol}. This is
  derived from \textbf{problems.csv}'s \textbf{Pars} and its adjacent
  columns
\item
  \texttt{NLStag}: a tag to classify the nls problem
\item
  \texttt{NLSref}: nls object created using \textbf{refsol} to a problem
\end{itemize}

If x in "\_x" is greater than 1, it means that it is a children test
file belonging to the same "problemName family of test files. It
contains a subset of the contents as mentioned above. For example,
\emph{NLSprob\_2.R} may contain just a new \texttt{NLSstart} or may
contain \texttt{NLSupper} and \texttt{NLSlower} and so on.

To add ``children'' scripts, the naming convention is compulsory.
``NLSproblemName.x'' must have the x value just succeeding the last x
value for the same \emph{problemName}.

\hypertarget{coverage}{%
\subsection{Coverage}\label{coverage}}

The idea of hierarchy in test files allows the diverse cases of:

\begin{itemize}
\tightlist
\item
  multiple starting values that may be acceptable or \textbf{infeasible}
  (i.e., outside constraints)
\item
  bounded problems having a good or \textbf{inadmissible} (hence silly)
  bounds where one or more lower bounds is greater than the
  corresponding upper bound. For such problems all other inputs are
  irrelevant. The problem is not well-posed.
\item
  multiple sets of weights for the same problem. Some of these may
  render a problem more or less difficult to solve.
\item
  ability to work with subsets of the problem's data. At the time of
  writing we have not pursued subsets beyond a few trial cases.
\end{itemize}

The \textbf{methods.csv} allows the possibility for adding more solvers
to the existing set:

\begin{itemize}
\tightlist
\item
  functions as \textbf{solver}s viz.~\texttt{nlsj::nlsj},
  \texttt{nlsr::nlxb} and \texttt{minpack.lm::nlsLM}
\item
  existing and new \textbf{algorithm}s viz.~``marquardt'', ``plinear'',
  ``port'' and ``Gauss-Newton'' and its variants
\end{itemize}

The \texttt{run()} function currently only compares essential estimation
quantities, that is the sum of squares and the parameters that generate
this value.

\hypertarget{package-output-files}{%
\subsection{Package Output Files}\label{package-output-files}}

\hypertarget{nlsdatabase.csv}{%
\subsubsection{nlsDatabase.csv}\label{nlsdatabase.csv}}

This csv file stores a final output of the \texttt{run()} function. It
has 11 columns -

\begin{itemize}
\tightlist
\item
  \texttt{DateTime}: the data and time (space-separated) the program
  runs a particular \textbf{FileName}
\item
  \texttt{MachID}: a one-line machine summary useful for characterizing
  different machines
\item
  \texttt{FileName}: the name of the file run inside the \texttt{run()}
  function. It is used from \textbf{Problems.csv}
\item
  \texttt{Solver}: the nls function used from \textbf{methods.csv} to
  solve a nonlinear problem
\item
  \texttt{Algorithm}: the algorithm used from \textbf{methods.csv} to
  solve a nonlinear problem
\item
  \texttt{Control}: the control settings used from \textbf{methods.csv}
  to solve a nonlinear problem
\item
  \texttt{Parameters}: result on comparing \textbf{Refsol} parameter
  values with that of \textbf{Solver}'s. It is \texttt{TRUE} if both are
  equal, a \texttt{NUMERIC} indicating a ``Mean relative difference''
  and \texttt{NA} if the \textbf{solver} fails to run.
\item
  \texttt{SSquares}: result on comparing \textbf{Refsol} sum of squares
  value with that of \textbf{Solver}'s. It is \texttt{TRUE} if both are
  equal, a \texttt{NUMERIC} indicating a ``Mean relative difference''
  and \texttt{NA} if the \textbf{solver} fails to run.
\item
  \texttt{Better}: comment based on \textbf{Parameters} and
  \textbf{SSquares}. It is \texttt{EQUAL} if both are \texttt{TRUE},
  \texttt{BETTER} if \emph{SSquares(Solver)} is less than
  \emph{SSquares(refsol)} and \texttt{Worse} otherwise.
\item
  \texttt{RefSol}: the package that gives the best parameter and sum of
  squares value for a problem. \texttt{NLStag} is used here.
\item
  \texttt{Tags}: the tag used to classifiy the problem. \texttt{NLStag}
  value is used here.
\end{itemize}

\hypertarget{nlserrorlog.csv}{%
\subsubsection{nlsErrorLog.csv}\label{nlserrorlog.csv}}

This csv file stores a final output of the \texttt{run()} function and
is essentially an error-log. Further, it is a subset of the
\textbf{nlsDatabase.csv} which stores information of only those files
which on which a particular \emph{solver} does not run. It has 7 columns
-

\begin{itemize}
\tightlist
\item
  \texttt{DateTime}: the data and time (space-separated) the program
  runs a particular \textbf{FileName}
\item
  \texttt{MachID}: a one-line machine summary useful for characterizing
  different machines
\item
  \texttt{FileName}: the name of the file run inside the \texttt{run()}
  function that observed an error.
\item
  \texttt{Solver}: the nls function used from \textbf{methods.csv} that
  caused the error in solving \textbf{FileName} problem
\item
  \texttt{Algorithm}: the algorithm used from \textbf{methods.csv} to
  solve the nonlinear problem
\item
  \texttt{Control}: the control settings used from \textbf{methods.csv}
  to solve the nonlinear problem
\item
  \texttt{Message}: the error message associated with the problem that
  comes from the \texttt{Solver}
\end{itemize}

\hypertarget{some-observations}{%
\subsection{Some observations}\label{some-observations}}

On using \textbf{nlsCompare}, we have made some observations that point
towards the usefulness of a package.

\begin{itemize}
\tightlist
\item
  \textbf{nlsDatabase.csv} gave us an example (Tetra\_1.R) where a
  solver indeed turned out to be better than our perceived reference
  solver. In this case, the problem has some quite extreme properties
  that were not obvious, and the output from \textbf{nlsCompare} is
  leading hopefully to some new functionality.
\item
  Sometimes the reference values, reported in decimal notation are not
  sufficiently precise and a ``better'' sum of squares is tagged. As we
  discover such ``better'' instances, we will deal with them on a case
  by case basis by replacing the references values appropriately.
\item
  \textbf{nlsDatabase.csv} shows which packages fail in a problem case
  and which succeed
\end{itemize}

\hypertarget{future-work}{%
\subsection{Future work}\label{future-work}}

A few things we would like to incorporate in \textbf{nlsCompare} -

\begin{itemize}
\tightlist
\item
  add other estimation quantities like \textbf{Convergence},
  \textbf{Jacobian}, and some characteristics of the solutions related
  to uncertainty in the parameters.
\item
  add \texttt{run\_specific} that runs a particular file or a list of
  files rather than all files as is listed in\\
  \textbf{problems.csv}
\item
  automatically update the reference values in the R scripts in
  ./nlsCompare/inst/scripts on encountering ``better'' values and use
  \textbf{pkgVersion} and\textbf{LastUpdated} columns in
  \textbf{problems.csv} to keep track of such changes.
\item
  add more robust bound checking mechanisms like \texttt{optimx::bmchk}
\item
  include more problems in the database; for example, from
  \textbf{NISTnls} or other collections in the literature.
\end{itemize}

\bibliography{RJreferences.bib}

\address{%
Arkajyoti Bhattacharjee\(^1\)\\
Mathematics and Statistics Department\\%
Indian Institute of Technology, Kanpur\\ India\\
%
%
%
\href{mailto:arkastat98@gmail.com}{\nolinkurl{arkastat98@gmail.com}}%
}

\address{%
John C. Nash\(^2\)\\
University of Ottawa\\%
\\
%
\url{https://journal.r-project.org}\\%
\textit{ORCiD: \href{https://orcid.org/0000-0002-9079-593X}{0000-0002-9079-593X}}\\%
\href{mailto:nashjc@uottawa.ca}{\nolinkurl{nashjc@uottawa.ca}}%
}

\address{%
Heather Turner\(^3\)\\
Department of Statistics\\%
University of Warwick\\ Coventry\\ United Kingdom\\
%
%
\textit{ORCiD: \href{https://orcid.org/0000-0002-1256-3375}{0000-0002-1256-3375}}\\%
\href{mailto:ht@heatherturner.net}{\nolinkurl{ht@heatherturner.net}}%
}
